\documentclass[a4paper, 12pt]{report}
\usepackage[utf8]{inputenc}
\usepackage[russian]{babel}
\usepackage{tabularx}
\usepackage{geometry}
\geometry{
	left=20mm,
	right=20mm,
	top=20mm,
	bottom=20mm
}

\begin{document}
\begin{titlepage}
	\centering
	{\scshape\Large Универзитет Унион \par}
	\vspace{1cm}
	{\scshape\LARGE Рачунарски факултет\par}
	\vspace{2cm}
	{\scshape\large Лазар Миленковић\par}
	\vspace{0.5cm}
	{\huge\bfseries Поређење похлепних алгоритама и динамичког програмирања за  стратегију игре Таблић и њених модификација \par}
	\vspace{2cm}
	\vfill

	
	% Bottom of the page
	{\large 10.07.2017\par}
\end{titlepage}
\pagebreak

\begin{tabularx}{\textwidth}{lX}
	Кандидат: & Лазар Миленковић  \\
	Број индекса: & РН 8/13 \\
	Наслов: & Поређење похлепних алгоритама и динамичког програмирања за  стратегију игре Таблић и њених модификација \\
	Ментор: & Др Владимир Миловановић\\
\end{tabularx}
\pagebreak
\section{Апстракт}
У овом раду пореде се временска, меморијска сложеност, као и ефикасност више похлепних стратегија, као и стратегија динамичког програмирања за игру Таблић и њених модификација. Таблић је популарна игра картама у којој сваки играч наизменично повлачи једну карту са циљем да на крају игре има највећи збир сакупљених карата. У стандардној верзији, играч зна само карте које му се тренутно налазе у руци као и све карте које су прошле до сада. За ову верзију игре имплементирано је неколико похлепних стратегија, као и стратегија динамичког програмирања. Модификована верзија подразумева да играч зна којом стратегијом играју његови противници, као и редослед карата до краја партије. За ову верзију су такође имплементиране похлепна стратегија, динамичка стратегија, као и бектрек стратегија која гарантовано налази најбољи низ потеза. 

\end{document}